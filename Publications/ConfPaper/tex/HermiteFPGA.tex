%%%%%%%%%%%%%%%%%%%%%%% file typeinst.tex %%%%%%%%%%%%%%%%%%%%%%%%%
%
% This is the LaTeX source for the instructions to authors using
% the LaTeX document class 'llncs.cls' for contributions to 
% the Lecture Notes in Computer Sciences series.
% http://www.springer.com/lncs       Springer Heidelberg 2006/05/04
%
% It may be used as a template for your own input - copy it 
% to a new file with a new name and use it as the basis
% for your article. 
%
%%%%%%%%%%%%%%%%%%%%%%%%%%%%%%%%%%%%%%%%%%%%%%%%%%%%%%%%%%%%%%%%%%%


\documentclass[runningheads]{llncs}

\usepackage{amssymb}
\setcounter{tocdepth}{3}
\usepackage{graphicx}
\graphicspath{{./FIGURES/}}

\usepackage{algorithm}
\usepackage{algorithmic}
\usepackage{tabularx}
\usepackage{multirow}
\usepackage{rotating}


\usepackage[latin1]{inputenc}
\usepackage{url}
\urldef{\mailsa}\path|kartik@ee.iitb.ac.in|
\urldef{\mailsb}\path|gabriel.caffarena@ceu.es|
\urldef{\mailsc}\path|madhav@ee.iitb.ac.in|
\newcommand{\keywords}[1]{\par\addvspace\baselineskip
\noindent\keywordname\enspace\ignorespaces#1}


\def\NoNumber#1{{\def\alglinenumber##1{}\State #1}\addtocounter{ALG@line}{-1}}

\begin{document}

\mainmatter  % start of an individual contribution

% first the title is needed
\title{Low-latency Hermite Polynomial Characterization of Heartbeats using a Field-Programmable Gate Array}

% a short form should be given in case it is too long for the running head 
\titlerunning{Low-latency Hermite Polynomial Characterization using an FPGA}


% the name(s) of the author(s) follow(s) next
%
% NB: Chinese authors should write their first names(s) in front of
% their surnames. This ensures that the names appear correctly in
% the running heads and the author index.
%
\author{Kartik Lakhotia\inst{1} \and
 Gabriel Caffarena\inst{2} \and 
 Madhav P. Desai \inst{1}}

% if the list of authors exceeds the space for a headline,
% an abbreviated author list is needed
\authorrunning{Kartik Lakhotia et al.}
% (feature abused for this document to repeat the title also on left hand pages)



% the affiliations are given next
\institute{Indian Institute of Technology (Bombay), \\
Powai, Mumbai 400076, \\
India\\
\and 
University CEU-San Pablo,\\
Urb. Monteprincipe, 28668, Madrid, Spain\\
%\mailsa\\
\mailsb\\
%\mailsd\\
\url{http://biolab.uspceu.com}\\
%\mailsc\\
}

%
% NB: a more complex sample for affiliations and the mapping to the
% corresponding authors can be found in the file "llncs.dem" 
% (search for the string "\mainmatter" where a contribution starts).
% "llncs.dem" accompanies the document class "llncs.cls".
%
\toctitle{Lecture Notes in Computer Science}
\tocauthor{Authors' Instructions}
\maketitle


\begin{abstract}
The characterization of ECG heartbeats is a computationally intensive problem, and
both off-line and on-line (real-time) solutions to this problem are of great interest.
In this paper, we consider the use of a field-programmable gate-array (FPGA) to solve
a critical component of this problem. We describe an implementation of
a best-fit Hermite approximation of a heartbeat using six Hermite polynomials.  
The implementation is generated using an algorithm-to-hardware compiler tool-chain
and the resulting hardware is characterized using an off-the-shelf FPGA card.
The single beat best-fit computation latency is under $0.5ms$ with a power dissipation of under 10 watts. 

\keywords{Hermite approximation, ECG, QRS, Arrhythmia,  FPGA, Parallelization}
\end{abstract}


\section{Introduction}

Automatic ECG analysis and characterization can help in 
identifying anomalies in a long-term ECG recording. 
In particular, the characterization of the QRS complex by means of Hermite functions 
seems to be a reliable mechanism  for automatic classification of heartbeats \cite{j:lagerholm00}. 
The main advantages seem to be the low sensitivity to noise and artifacts, and the
compactness of the representation (e.g. a 144-sample QRS can be characterized with 7 parameters \cite{c:marquez13}). 
These advantages have made the Hermite representation a very common tool for characterizing the 
morphology of the beats \cite{j:lagerholm00,c:marquez13,c:braccini97,j:linh03a,j:linh03b}. 

ECG analysis using Hermite functions has a
substantial amount of parallelism.  Solutions to the problem have been investigated
using processors (and multi-cores) and graphics processing units (GPU's). 
In this paper, we consider the alternative route of using an FPGA to implement
the computations.  In particular, our work is motivated by the potential
of an FPGA (or eventually, a dedicated application-specific circuit) for low-latency
energy efficient heart-beat analysis.  

In generating the hardware for heart-beat analysis, we make extensive use of
algorithm-to-hardware techniques.  By this we mean that the hardware is
generated from an algorithmic specification that is written in a
high-level programming language ({\bf C} in this case), which is then
transformed to a circuit implementation using a set of compiler tools \cite{c:ahir}.
The resulting hardware is then mapped to an FPGA card (the ML605 card from Xilinx,
which uses a Virtex-6 FPGA).  The circuit is then exercised through the PCI-express 
interface and used to classify beats.  The round-trip latency of a single
beat classification was found to be under $0.5ms$.

%\begin{itemize}
%\item The assessment of the capabilities of GPUs for the off-line characterization of heartbeats using Hermite functions approximations
%\item The assessment of the capabilities of GPUs for the on-line characterization of heartbeats using Hermite functions approximations
%\end{itemize}

\section{QRS approximation by means of Hermite polynomials}\label{s:arr}

The aim of using the Hermite approximation to estimate heartbeats is to 
reduce the number of dimensions required to carry out the ECG classification, 
without sacrificing accuracy. 
The benchmarks used in this work come from the MIT-BIH arrhythmia database \cite{j:moody01} which is made up of 
48 ECG recordings whose beats  have been manually annotated by two cardiologists. Each file from the database 
contains 2 ECG channels, sampled at a frequency of 360 Hz and with a duration of approximately 2000 beats. 
In particular, here we are addressing the characterization of the morphology of the QRS complexes since this 
morphology, together with the distance between each pair of consecutive heartbeats, permits the identification of the majority of arrhythmias.

Firstly,  the ECG files are preprocessed to remove baseline drift. Secondly, the QRS complexes for each heartbeat are extracted by finding the peak of the beat (e.g. the R wave) and selecting a  window of 200 ms centered on the heartbeat. Given that all the Hermite functions converge to zero both in $t=\infty$ and $t=-\infty$, the original QRS signal is extended to 400 ms by adding 100-ms sequences of zeros at each side of the complex. Thus, the QRS data are stored in a 144-sample vector $\vec{x}=\{x(t)\}$. This vector can be estimated with a linear combination of $N$ Hermite basis functions

\begin{equation}\label{eqn:hat}
\hat{x}(t)=\sum_{n=0}^{N-1}c_n(\sigma )\phi_n(t,\sigma),
\end{equation}

\noindent with

\begin{equation}\label{eqn:phi}
\phi_n(t,\sigma )=\frac{1}{\sqrt{\sigma 2^n n!\sqrt{\pi}}}e^{-t^2/2\sigma^2}H_n(t/\sigma) 
\end{equation}

\noindent being $H(t/\sigma)$ the Hermite polynomials. These polynomials can be computed recursively as

\begin{equation}
H_n(x)=2xH_{n-1}(x)-2(n-1)H_{n-2}(x),
\end{equation}

\noindent where $H_0(x)=1$ and $H_1(x)=2x$.

The parameter $\sigma$ controls the width of the polynomials. In \cite{j:lagerholm00} the maximum value 
of $\sigma$ for a given order $n$ is estimated.  As the value of $n$ increases, the value of $\sigma_{MAX}$ decreases.

The optimal coefficients that minimize the estimation error for a given $\sigma$ are

\begin{equation}\label{eqn:c}
c_n(\sigma)=\sum_{t} x(t)\cdot \phi_n(t,\sigma) \textrm{~\cite{j:lagerholm00}.}
\end{equation}

Once the suitable set of $\sigma$ and $\vec{c}=\{c_n(\sigma)\}$  \mbox{$(n\in [0,N-1])$} are found for each heartbeat, 
it is possible to use only these figures to perform morphological classification of the heartbeats. 

\section{Beginning the FPGA implementation: the algorithm}

The algorithm used in the FPGA implementation is as follows: the 
implementation first receives the values of the Hermite polynomial basis
functions, and  stores them in distinct arrays in the hardware.  Distinct
basic functions are needed for each $n$ and $\sigma$.  The current implementation
uses six values of $n$ (from $0$ to $5$) and ten values of $\sigma$.

After this initialization step, the hardware listens for heart beats. When
a complete heart-beat (144 samples) is received, the inner products of the
heart-beat with all the basic functions is calculated in a double loop.  
After all inner products are calculated, the inner product coefficients
are used to compute the best fit among the different values of $\sigma$.
The best-fit $\sigma$ index and the fitted values are then written
out of the hardware.

\begin{verbatim}
// Hardware engine algorithm (Daemon)
void HermiteBestFit()
{  
    // stored in 6 distinct arrays
    // hF0,hF1,.. hF5.  hFn stores
    // all the basic functions for 
    // order n (for different values
    // of sigma).
    receiveHermiteBasisFunctions();

    while(1)
    {
        // received in 144 entry
        // double precision vector.
        receiveHeartBeat();
        // compute inner products
        // with all basis functions
        // (across n, sigma).
        innerProducts();
        // 
        // best fit sigma
        //
        findBestFit();
        //
        // report results
        //
        reportResults();
    }
}
\end{verbatim}

The algorithm as described above is purely sequential
and does not contain any explicit parallelization.  The AHIR compiler
is intelligent enough to extract parallelism from the two critical
loops (in the inner-product and best-fit functions).

Even with this simple coding of the hardware algorithm,
we observe that excellent real-time performance
is observed (in comparison with CPU/GPU implementations).
Going further, it is possible to specify explicit parallelism by
writing the processing as a two step pipeline consisting
of separate threads for inner-product and best-fit
computations  These investigations are ongoing.



\subsection{The inner product loop}

The inner product loop can be described as follows:
\begin{verbatim}
void  innerProduct()
{
  int I;
  for (I=0; I < NSAMPLES; I++)
  {
     double x = inputData[I];
     for(SI = 0; SI < NSIGMAS; SI++)
     {
        int I0 = I + Offset[SI];
        double p0 = (x0*hF0[I0]);
        double p1 = (x0*hF1[I0]);
        double p2 = (x0*hF2[I0]);
        double p3 = (x0*hF3[I0]);
        double p4 = (x0*hF4[I0]);
        double p5 = (x0*hF5[I0]);
        dotP0[SI] += p0;
        dotP1[SI] += p1;
        dotP2[SI] += p2;
        dotP3[SI] += p3;
        dotP4[SI] += p4;
        dotP5[SI] += p5;
     }
}
\end{verbatim}
The outer loop is over the samples, and the inner
loop across the $\sigma$ values.  There is a high-level
of parallelism in the inner loop which can be further
boosted by unrolling the outer loop.   The AHIR compiler
implements this entire function using a single double-precision
multiplier and a single double-precision adder.  Further
note that the arrays $hFn$ and $dotPn$ are declared on
a per-$n$ basis. This allows the AHIR compiler to
map the arrays to distinct memory spaces, thus increasing
the memory access bandwidth in the hardware.

\subsection{The minimum-mean-square loop}

This loop is also quite straightforward.
\begin{verbatim}
void computeMSE()
{
  int I, SI;
  best_mse = 1.0e+20;
  best_sigma_index = -1;
  for (I=0; I<NSAMPLES; I=I+4)
  {
     for (SI=0; SI<NSIGMAS; SI++)
     {
        int fetchIndex0 = I + Offset[SI]; 
        double p0 = (dotP0[SI]*hF0[fetchIndex0]);
        double p1 = (dotP1[SI]*hF1[fetchIndex0]);
        double p2 = (dotP2[SI]*hF2[fetchIndex0]);
        double p3 = (dotP3[SI]*hF3[fetchIndex0]);
        double p4 = (dotP4[SI]*hF4[fetchIndex0]);
        double p5 = (dotP5[SI]*hF5[fetchIndex0]);
        double diff = (inputData[I]-
                         ((p0+p1) + (p2+p3) + (p4+p5)));
        err[SI] += (diff*diff);
      }
    }
    for (SI=0; SI<NSIGMAS; SI++)
    {
        if(err[SI] <  best_mse)
        {
           best_mse = err[SI];
           best_sigma_index = SI;
        }
    }
}
\end{verbatim}


\subsection{Further optimizations}

The current implementation uses a simple sequential specification.
Further optimizations include: loop-unrolling, explicit pipelining,
and the use of multiple floating point operators.  All these optimizations
can be explored entirely at the algorithmic level using the AHIR
compiler tools.

\section{Hardware Implementation Details}\label{s:results}
The overall system has 3 major components : Host Computer (for S/W calculations), 
Communication infrastructure and H/W for Hermite Coefficient Computation.


We have used Xilinx ML605 card which features a Virtex-6 FPGA and 8-lane PCI express. 
THe host computes values of Hermite polynomials, reads Heartbeat samples 
from a file and sends them to FPGA. H/W running on FPGA then calculates \& selects 
Hermite coefficients which give best fit to input beat and reports back to Host. 


HDL for this design is generated using AHIR HLS toolchain. It is equipped with a library 
of heavily pipelined Floating Point operators and Loop Pipelining mechanism which 
enables the final VHDL to extract parallelism in C program. It uses pipes to communicate 
between testbench and the program, which translate to FIFOs in hardware. A simple integration 
interface between these FIFOs and RIFFA channels completes the system. 


The RIFFA software drivers and interface infrastructure is used to communicate between 
host and FPGA card \cite{j:jacobsen13}. It supports upto 12 independent
channels for data transmission. All of them end in separate Rx/Tx FIFOs on FPGA that can operate on 
different clock domains at either ends. In our design, PCIe core operates at 250MHz 
while the ECG hardware operation and FIFO accesses are at 100MHz. A simple integration 
interface bridges RIFFA and AHIR generated FIFOs.
By incorporating functions provided by RIFFA driver, same testbench used for 
Software verification can be used for verifying the hardware.
%  Describe the 
%    - overall system.
%    - card
%    - the driver
%    - the hardware generation process.
%
\section{Results}\label{s:results}

We calculate round-trip delay and FPGA core power consumption for processing single beat. 
The round-trip delay is the time interval between the beginning of transmission of 
beat-data from the host to the hardware and the beginning of reception of best fit
coefficients from the hardware. We also report the Device Utilization trends observed
with 2-way and 4-way loop unrolling.

\begin{table}[ht]
\caption{} %title
\centering
\begin{tabular}{c @{\hskip 0.06in}| @{\hskip 0.06in}c@{\hskip 0.2in}c @{\hskip 0.2in} c @{\hskip 0.2in} c} %5 centered columns
\hline\hline\\[-1.5ex]
Code Optimzation & \begin{tabular}[c]{@{}c@{}}Slice LUT  \\ Utilization  \end{tabular}& \begin{tabular}[c]{@{}c@{}}Slice Register  \\Utilization  \end{tabular}& \begin{tabular}[c]{@{}c@{}}Processing  \\Latency \end{tabular} & \begin{tabular}[c]{@{}c@{}}FPGA core Power \\ Consumption \end{tabular}\\ [2ex] \hline \\[-1.5ex]%heading
No Unrolling & 56839 & 65995 & 1.39ms & 2.75W \\
2-way Unrolled & 65895 & 80709 & 0.80ms & 2.88W \\
4-way Unrolled & 84331 & 110165 & 0.44ms & 3.09W \\ [1ex]
\hline
\end{tabular}
\end{table}



\begin{verbatim}
Minimum latency achieved with Four-way-unrolling = 0.39 ms
\end{verbatim}
The observed power dissipation is 3W.  The hardware utilization
in 4-way unrolled system is less than $55\%$ of the FPGA resource.

\subsection{Comparison with GPU/CPU implementations}

\section{Conclusions}\label{s:conclusions}

% Bibliography
\bibliographystyle{splncs}
\bibliography{refsQRS}

\end{document}
