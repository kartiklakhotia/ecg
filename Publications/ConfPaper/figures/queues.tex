% Tikz-code with queue shapes.
% Author: Ruslan Krenzler.
% 19 Juli 2013.
% Warning: the coder, has almost no expirience 
% in TeX and Tikz-programming.
% Auchtung: Der Programmierer, hat kaum Erfahrung mit TeX- und
% Tikz-Programmierung. 

% See Tikz-Manual for programing custom shapes.

\usepackage{pgfplots}
\usepackage{pgf}
\usetikzlibrary{shapes,arrows} % Rounded rectangle for the server.

\usepackage{ifthen}

% We define some parameters for waiting queue symbols.
\makeatletter
\pgfkeys{/tikz/queue head/.initial=east}
\pgfkeys{/tikz/queue size/.initial=infinite}

% The queue symbols are similar to rectangle,
% drawed in a different way.

\pgfdeclareshape{queue}
{
%%% Inherit anchors from rectangle %%%.
	\inheritsavedanchors[from={rectangle}]
	\inheritanchorborder[from={rectangle}]
	\inheritanchor[from={rectangle}]{center}
	\inheritanchor[from={rectangle}]{north}
	\inheritanchor[from={rectangle}]{east}
	\inheritanchor[from={rectangle}]{south}
	\inheritanchor[from={rectangle}]{west}
	
	
%%% Set some auxilirary variables for drawing. %%%.
	
	\savedmacro{\queuehead}{\pgfkeysvalueof{/tikz/queue head}}

	\saveddimen{\width}{
		\pgf@x=\wd\pgfnodeparttextbox
		% Correct dimension if it is larger than the minimum width.
		% Convert minimum width to dimension.
		\pgfmathsetlength{\pgf@xa}{\pgfkeysvalueof{/pgf/minimum width}}%
		\ifdim\pgf@x<\pgf@xa
			\pgf@x=\pgf@xa
		\fi
	}
	\saveddimen{\halfwidth}{
		\pgf@x=\wd\pgfnodeparttextbox
		% Correct dimension if it is larger than the minimum width.
		% Convert minimum width to dimension.
		\pgfmathsetlength{\pgf@xa}{\pgfkeysvalueof{/pgf/minimum width}}%
		\ifdim\pgf@x<\pgf@xa
			\pgf@x=\pgf@xa
		\fi
		\divide \pgf@x by 2
	}
	\saveddimen{\height}{
		\pgf@x=\ht\pgfnodeparttextbox
		% Correct dimension if it is larger than the minimum height.
		% Convert minimum height to dimension.
		\pgfmathsetlength{\pgf@xa}{\pgfkeysvalueof{/pgf/minimum height}}%
		\ifdim\pgf@x<\pgf@xa
			\pgf@x=\pgf@xa
		\fi
	}
	\saveddimen{\halfheight}{
		\pgf@x=\ht\pgfnodeparttextbox
		% Correct dimension if it is larger than the minimum height.
		% Convert minimum height to dimension.
		\pgfmathsetlength{\pgf@xa}{\pgfkeysvalueof{/pgf/minimum height}}%
		\ifdim\pgf@x<\pgf@xa
			\pgf@x=\pgf@xa
		\fi
		\divide \pgf@x by 2
	}
	
%%% Define head and tail anchors. %%%
	
		% For some reasons I cannot just reaturn ihnerit east, west, a.s.o
	% anchours. Either there are problems or the result is strange.
	% Calculate head anchors manually.
	
	\anchor{head}{
		\ifthenelse{\equal{\pgfkeysvalueof{/tikz/queue head}}{east}}{
				\northeast
				\advance \pgf@y by -\halfheight
		}{}
		\ifthenelse{\equal{\pgfkeysvalueof{/tikz/queue head}}{west}}{
				\southwest
				\advance \pgf@y by \halfheight
		}{}
		\ifthenelse{\equal{\pgfkeysvalueof{/tikz/queue head}}{north}}{
				\northeast
				\advance \pgf@x by -\halfwidth
		}{}
		\ifthenelse{\equal{\pgfkeysvalueof{/tikz/queue head}}{south}}{
				\southwest
				\advance \pgf@x by \halfwidth
		}{}
	}
	
	% The tail lies oposite to the head.
	\anchor{tail}{
		\ifthenelse{\equal{\pgfkeysvalueof{/tikz/queue head}}{east}}{
				\southwest
				\advance \pgf@y by \halfheight
		}{}
		\ifthenelse{\equal{\pgfkeysvalueof{/tikz/queue head}}{west}}{
				\northeast
				\advance \pgf@y by -\halfheight
		}{}
		\ifthenelse{\equal{\pgfkeysvalueof{/tikz/queue head}}{north}}{
				\southwest
				\advance \pgf@x by \halfwidth
		}{}
		\ifthenelse{\equal{\pgfkeysvalueof{/tikz/queue head}}{south}}{
				\northeast
				\advance \pgf@x by -\halfwidth
		}{}
	}

%%% Draw  the /M/M/1/\infty queue.
	\backgroundpath{
		% Save coordinates of the bottom left and top right edges
		\pgfextractx\pgf@xa{\southwest}
		\pgfextracty\pgf@ya{\southwest}
		\pgfextractx\pgf@xb{\northeast}
		\pgfextracty\pgf@yb{\northeast}
		
		% Determine if the queue (symbol) is vertical.
		% That is with the head on east or west side).
 		\newboolean{vertical}
		\ifthenelse{\equal{\pgfkeysvalueof{/tikz/queue head}}{north}
		 	\OR  
			\equal{\pgfkeysvalueof{/tikz/queue head}}{south}}{
			\setboolean{vertical}{true}	
		}{
			\setboolean{vertical}{false}
		}
		
		% Draw the left and the right lines for a vertical queue
		% and the bottom and the upper lines for a horizontal queue.
		\ifthenelse{\boolean{vertical}}{
			% Draw left and right lines.
			\pgfpathmoveto{\pgfpoint{\pgf@xa}{\pgf@ya}}
			\pgfpathlineto{\pgfpoint{\pgf@xa}{\pgf@yb}}
			\pgfpathmoveto{\pgfpoint{\pgf@xb}{\pgf@ya}}
			\pgfpathlineto{\pgfpoint{\pgf@xb}{\pgf@yb}}	
		}{
		% Otherwise it is a horizontal queue.
			\pgfpathmoveto{\pgfpoint{\pgf@xa}{\pgf@ya}}
			\pgfpathlineto{\pgfpoint{\pgf@xb}{\pgf@ya}}
			\pgfpathmoveto{\pgfpoint{\pgf@xa}{\pgf@yb}}
			\pgfpathlineto{\pgfpoint{\pgf@xb}{\pgf@yb}}		
		}
		
		% Define some auxilirary variables
		\newdimen \midy
		\midy=\pgf@ya
		\advance \midy by \pgf@yb
		\divide \midy by 2
		
		\newdimen \midx
		\midx=\pgf@xa
		\advance \midx by \pgf@xb
		\divide \midx by 2
		
		% Draw the remaining queue parts dependent on 
		% the queue orientation (head position) and
		% queue type (open, closed).
		
		\ifthenelse{\boolean{vertical}}{
			% Draw vertical queues.
			\newdimen \cellheight
			\cellheight=\height
			\divide\cellheight by 6
			\newdimen \currcelly
			\currcelly=\pgf@ya
			
			\ifthenelse{\equal{\pgfkeysvalueof{/tikz/queue size}}{infinite}
			 		\AND
			 		\equal{\pgfkeysvalueof{/tikz/queue head}}{north}}
        	{
       		 	% Draw cells
       	 		\foreach \y in {2,3,4,5,6}{
					\advance \currcelly by \y\cellheight
					\pgfpathmoveto{\pgfpoint{\pgf@xa}{\currcelly}}
					\pgfpathlineto{\pgfpoint{\pgf@xb}{\currcelly}}
				}
				% Draw 3 points.
				\pgf@yc = \pgf@ya
				\advance \pgf@yc by \cellheight
				\filldraw (\midx, \pgf@yc) circle (1pt);
				\filldraw (\midx, \pgf@yc+5pt) circle (1pt);
				\filldraw (\midx, \pgf@yc-5pt) circle (1pt);
       	 	}{}

			 \ifthenelse{\equal{\pgfkeysvalueof{/tikz/queue size}}{infinite}
			 		\AND
			 		\equal{\pgfkeysvalueof{/tikz/queue head}}{south}}
       	 	{
       		 	% Draw cells
       	 		\foreach \y in {0,1,2,3,4}{
					\advance \currcelly by \y\cellheight
					\pgfpathmoveto{\pgfpoint{\pgf@xa}{\currcelly}}
					\pgfpathlineto{\pgfpoint{\pgf@xb}{\currcelly}}
				}
				% Draw 3 points.
				\pgf@yc = \pgf@yb
				\advance \pgf@yc by -\cellheight
				\filldraw (\midx, \pgf@yc) circle (1pt);
				\filldraw (\midx, \pgf@yc+5pt) circle (1pt);
				\filldraw (\midx, \pgf@yc-5pt) circle (1pt);
        	}{}
        	
        	\ifthenelse{\equal{\pgfkeysvalueof{/tikz/queue size}}{bounded}}
       	 	{
       		 	% Draw cells
       	 		\foreach \y in {0,1,2,4,5,6}{
					\advance \currcelly by \y\cellheight
					\pgfpathmoveto{\pgfpoint{\pgf@xa}{\currcelly}}
					\pgfpathlineto{\pgfpoint{\pgf@xb}{\currcelly}}
				}
				% Draw 3 points.
				\pgf@yc = \pgf@ya
				\advance \pgf@yc by 3\cellheight
				\filldraw (\midx, \pgf@yc) circle (1pt);
				\filldraw (\midx, \pgf@yc+5pt) circle (1pt);
				\filldraw (\midx, \pgf@yc-5pt) circle (1pt);
        	}{}
		}{
			% Draw horizontal queues.
			\newdimen \cellwidth
			\cellwidth=\width
			\divide\cellwidth by 6
			\newdimen \currcellx
			\currcellx=\pgf@xa
			\ifthenelse{\equal{\pgfkeysvalueof{/tikz/queue size}}{infinite}
			 		\AND
			 		\equal{\pgfkeysvalueof{/tikz/queue head}}{east}}
        	{
       		 	% Draw cells
       	 		\foreach \x in {2,3,4,5,6}{
					\advance \currcellx by \x\cellwidth
					\pgfpathmoveto{\pgfpoint{\currcellx}{\pgf@ya}}
					\pgfpathlineto{\pgfpoint{\currcellx}{\pgf@yb}}
				}
				% Draw 3 points.
				\pgf@xc = \pgf@xa
				\advance \pgf@xc by \cellwidth
				\filldraw (\pgf@xc,\midy) circle (1pt);
				\filldraw (\pgf@xc+5pt,\midy) circle (1pt);
				\filldraw (\pgf@xc-5pt,\midy) circle (1pt);
       	 	}{}

			 \ifthenelse{\equal{\pgfkeysvalueof{/tikz/queue size}}{infinite}
			 		\AND
			 		\equal{\pgfkeysvalueof{/tikz/queue head}}{west}}
       	 	{
	        	\foreach \x in {0,1,2,3,4}{
					\advance \currcellx by \x\cellwidth
					\pgfpathmoveto{\pgfpoint{\currcellx}{\pgf@ya}}
					\pgfpathlineto{\pgfpoint{\currcellx}{\pgf@yb}}
				}
				% Draw 3 points.
				\pgf@xc = \pgf@xb
				\advance \pgf@xc by -\cellwidth
				\filldraw (\pgf@xc,\midy) circle (1pt);
				\filldraw (\pgf@xc+5pt,\midy) circle (1pt);
				\filldraw (\pgf@xc-5pt,\midy) circle (1pt);
        	}{}
        	
        	\ifthenelse{\equal{\pgfkeysvalueof{/tikz/queue size}}{bounded}}
       	 	{
	        	\foreach \x in {0,1,2,4,5,6}{
					\advance \currcellx by \x\cellwidth
					\pgfpathmoveto{\pgfpoint{\currcellx}{\pgf@ya}}
					\pgfpathlineto{\pgfpoint{\currcellx}{\pgf@yb}}
				}
				% Draw 3 points.
				\pgf@xc = \pgf@xa
				\advance \pgf@xc by 3\cellwidth
				\filldraw (\pgf@xc,\midy) circle (1pt);
				\filldraw (\pgf@xc+5pt,\midy) circle (1pt);
				\filldraw (\pgf@xc-5pt,\midy) circle (1pt);
        	}{}
        } 
	}
}
