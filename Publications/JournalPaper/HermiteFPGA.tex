\documentclass[12pt]{article}
\usepackage{graphicx}
\hyphenation{op-tical net-works semi-conduc-tor}
\graphicspath{{./figures/}}


\title{A low-latency, low-power FPGA implementation of ECG signal characterization using Hermite polynomials}


%
% paper title
% can use linebreaks \\ within to get better formatting as desired

% author names and affiliations
% use a multiple column layout for up to three different
% affiliations
\author{Madhav Desai, Kartik Lakhotia \\
Indian Institute of Technology (Bombay)\\
Mumbai, India \\
Gabriel Caffarena \\
CEU San Pablo\\
Madrid, Spain}

\begin{document}
\maketitle


\begin{abstract}
ECG signal characterization is of critical importance in patient monitoring
and diagnosis.  This characterization is computationally intensive, and
low-power, on-line (real-time) solutions to this problem are of great interest.
In this paper, we present a dedicated hardware implementation of the ECG signal
chain.  Starting from 12-bit ADC samples of the ECG signal, the hardware implements
filtering, peak and QRS detection and least-squares Hermite polynomial fit on the 
heart-beats.  The hardware implementation has been validated on a Field Programmable 
Gate Array (FPGA).  The implementation is generated using an algorithm-to-hardware 
compiler tool-chain and the resulting hardware is characterized using an off-the-shelf 
FPGA card.  
The hardware complexity of the entire system is XXXX LUTs and YYYY flip-flops.
The single beat best-fit computation latency when using six Hermite basis
polynomials is under $0.5ms$ with an average power dissipation of less than
30mW, demonstrating true real-time applicability.  
\end{abstract}

\include{Body}

\end{document}



